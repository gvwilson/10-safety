\documentclass[10pt,letterpaper]{article}
%DIF LATEXDIFF DIFFERENCE FILE
%DIF DEL submission-01.tex   Tue Nov  3 11:03:18 2020
%DIF ADD 10-safety.tex       Tue Nov  3 11:01:35 2020
\input{settings}
%DIF < 
%DIF < \newcommand{\rulemajor}[1]{\section*{#1}}
%DIF < \newcommand{\withurl}[2]{{#1}\footnote{{\texttt{#2}}}}
%DIF PREAMBLE EXTENSION ADDED BY LATEXDIFF
%DIF UNDERLINE PREAMBLE %DIF PREAMBLE
\RequirePackage[normalem]{ulem} %DIF PREAMBLE
\RequirePackage{color}\definecolor{RED}{rgb}{1,0,0}\definecolor{BLUE}{rgb}{0,0,1} %DIF PREAMBLE
\providecommand{\DIFadd}[1]{{\protect\color{blue}\uwave{#1}}} %DIF PREAMBLE
\providecommand{\DIFdel}[1]{{\protect\color{red}\sout{#1}}}                      %DIF PREAMBLE
%DIF SAFE PREAMBLE %DIF PREAMBLE
\providecommand{\DIFaddbegin}{} %DIF PREAMBLE
\providecommand{\DIFaddend}{} %DIF PREAMBLE
\providecommand{\DIFdelbegin}{} %DIF PREAMBLE
\providecommand{\DIFdelend}{} %DIF PREAMBLE
\providecommand{\DIFmodbegin}{} %DIF PREAMBLE
\providecommand{\DIFmodend}{} %DIF PREAMBLE
%DIF FLOATSAFE PREAMBLE %DIF PREAMBLE
\providecommand{\DIFaddFL}[1]{\DIFadd{#1}} %DIF PREAMBLE
\providecommand{\DIFdelFL}[1]{\DIFdel{#1}} %DIF PREAMBLE
\providecommand{\DIFaddbeginFL}{} %DIF PREAMBLE
\providecommand{\DIFaddendFL}{} %DIF PREAMBLE
\providecommand{\DIFdelbeginFL}{} %DIF PREAMBLE
\providecommand{\DIFdelendFL}{} %DIF PREAMBLE
%DIF LISTINGS PREAMBLE %DIF PREAMBLE
\RequirePackage{listings} %DIF PREAMBLE
\RequirePackage{color} %DIF PREAMBLE
\lstdefinelanguage{DIFcode}{ %DIF PREAMBLE
%DIF DIFCODE_UNDERLINE %DIF PREAMBLE
  moredelim=[il][\color{red}\sout]{\%DIF\ <\ }, %DIF PREAMBLE
  moredelim=[il][\color{blue}\uwave]{\%DIF\ >\ } %DIF PREAMBLE
} %DIF PREAMBLE
\lstdefinestyle{DIFverbatimstyle}{ %DIF PREAMBLE
	language=DIFcode, %DIF PREAMBLE
	basicstyle=\ttfamily, %DIF PREAMBLE
	columns=fullflexible, %DIF PREAMBLE
	keepspaces=true %DIF PREAMBLE
} %DIF PREAMBLE
\lstnewenvironment{DIFverbatim}{\lstset{style=DIFverbatimstyle}}{} %DIF PREAMBLE
\lstnewenvironment{DIFverbatim*}{\lstset{style=DIFverbatimstyle,showspaces=true}}{} %DIF PREAMBLE
%DIF END PREAMBLE EXTENSION ADDED BY LATEXDIFF

\begin{document}
\vspace*{0.2in}

\begin{flushleft}
{\Large
\textbf\newline{Ten Quick Tips for Staying Safe Online}
}
\newline
\\
{\DIFaddbegin \DIFadd{Danielle Smalls}}\DIFadd{\textsuperscript{1{\ddag}*}
}{\DIFaddend Greg Wilson}\textsuperscript{\DIFdelbegin \DIFdel{1}\DIFdelend \DIFaddbegin \DIFadd{2}\DIFaddend {\ddag}\DIFdelbegin \DIFdel{*}\DIFdelend }
\\
\bigskip
\textbf{1} \DIFdelbegin \DIFdel{RStudio, Inc., Toronto, ON, Canada}\DIFdelend \DIFaddbegin \DIFadd{MIR Community Group}\DIFaddend \\
\DIFaddbegin \textbf{\DIFadd{2}} \DIFadd{RStudio PBC}\\
\DIFaddend * Corresponding author, \DIFdelbegin \DIFdel{greg.wilson}\DIFdelend \DIFaddbegin \DIFadd{smalls.danielle}\DIFaddend @\DIFdelbegin \DIFdel{rstudio}\DIFdelend \DIFaddbegin \DIFadd{gmail}\DIFaddend .com. \\
\bigskip
{\ddag} These authors contributed equally to this work.
\end{flushleft}

\section*{Introduction}

Researchers studying everything from sexual health to COVID-19 to gun violence
are increasingly likely to be targeted because of their work.  While research
institutions have rules and guidelines for safeguarding sensitive information,
these usually do not address the problem of keeping \emph{individuals} safe from
either targeted attacks like Climategate \cite{Natu2010} or the kinds of
``drive-by'' threats that everyone now faces regardless of their occupation.

Hollywood depictions of everyday threats are as far from reality as their
portrayals of scientists, but more realistic guidance for personal digital
security is now freely available \cite{FLD,EFJ2015,EFF}. The ten quick tips in
this paper are a starting point: while they apply to everyone, they \DIFdelbegin \DIFdel{are particularly relevant to researchers .
They }\DIFdelend \DIFaddbegin \DIFadd{were
developed with researchers in mind. While researchers expect their work to be
scrutinized by the academic community, they should not expect to endure
harassment due to the visibility of their published works. These rules }\DIFaddend do not
guarantee complete safety, any more than seatbelts guarantee safe driving, but
following them greatly reduces the likelihood of harm.

\DIFdelbegin %DIFDELCMD < \rulemajor{Rule 1: Put on your own mask.}
%DIFDELCMD < %%%
\DIFdelend \DIFaddbegin \section*{\DIFadd{Rule 1: Put on your own mask.}}
\DIFaddend 

The first and most important rule is that we should not rely on companies,
universities, and other institutions to protect us, for the simple reason that
they are not penalized if they don't. As recently as ten years ago we could
blame the lack of meaningful institutional liability for data breaches on the
law being slow to catch up with rapidly-changing technology. \DIFdelbegin \DIFdel{Today,
it's clear that the law hasn't caught up because large tech companies don't want it to.
Weekly reports of }\DIFdelend \DIFaddbegin \DIFadd{Accountability for
these breaches is practically non-existent: }\DIFaddend data breaches have minimal impact on
\DIFdelbegin \DIFdel{profitability and no individual ever goes to jail
}\DIFdelend \DIFaddbegin \DIFadd{companies' profitability and individuals are almost never fined, much less
jailed.
}\DIFaddend 

Much of what institutions force us to go through \DIFdelbegin \DIFdel{online }\DIFdelend is \textbf{security theater}
intended to make us believe something is being done rather than to actually make
us safer. Requiring people to take off their shoes at airports is \DIFdelbegin \DIFdel{perhaps the best-known }\DIFdelend \DIFaddbegin \DIFadd{one }\DIFaddend example;
random searches of backpacks and bags at the entrance to the subway is another,
since it's hard to imagine that a would-be attacker \emph{wouldn't} just \DIFaddbegin \DIFadd{go }\DIFaddend to
another entrance. \DIFaddbegin \DIFadd{(Bruce Schneier's blog
has many examples of
security theater and the harm it does.)
}\DIFaddend 

Security theater is counter-productive because it encourages us to cut corners
in ways that actually make us \emph{less} safe. For example, forcing people to
change passwords every three months encourages people to choose memorable (and
therefore easy-to-guess) passwords.

\DIFdelbegin %DIFDELCMD < \rulemajor{Rule 2: Digital security is rarely the weakest link.}
%DIFDELCMD < %%%
\DIFdelend \DIFaddbegin \section*{\DIFadd{Rule 2: Digital security is rarely the weakest link.}}
\DIFaddend 

The second rule is to remember that most attacks take place offline, and that
the most effective ones are often the simplest. At an airport several years ago,
\DIFdelbegin \DIFdel{the }\DIFdelend \DIFaddbegin \DIFadd{one }\DIFaddend author heard a professor of computer science try to \DIFdelbegin \DIFdel{re-set }\DIFdelend \DIFaddbegin \DIFadd{reset }\DIFaddend an online account
over the phone. In just a couple of minutes, they had inadvertently told
everyone in the lounge their full name, their date of birth, the three-digit
verification code on the back of their credit card, and what was almost
certainly their mother's maiden name.

\DIFdelbegin \DIFdel{On that same trip,
a complete stranger (also a computer science professor)
asked the author to watch their laptop while they went to the washroom.
They were logged in to several online accounts at the time,
and even if they weren't,
the passwords saved in their browser could have been viewed with just a few mouse clicks.
}%DIFDELCMD < 

%DIFDELCMD < %%%
\DIFdelend The moral of \DIFdelbegin \DIFdel{these stories }\DIFdelend \DIFaddbegin \DIFadd{this story }\DIFaddend is that safety comes from good habits, not technology.
\textbf{Social engineering} is far more common than hacking: in practice it is
far easier to trick someone into giving you their password than to break into
their devices digitally.

The key practice is \textbf{situational awareness}, which is a fancy way of
saying, ``Pay attention to what's happening and respond accordingly.'' \DIFdelbegin \DIFdel{You shouldn't always be afraid to walk after dark,
but you
should use good judgment about where not to go.
Similarly,
if you notice unusual activity or if you }\DIFdelend \DIFaddbegin \DIFadd{If you
}\DIFaddend start working on \DIFdelbegin \DIFdel{something that may attract unwanted attention,
}\DIFdelend \DIFaddbegin \DIFadd{a high-profile subject that will likely attract controversy }\DIFaddend you
should take more precautions than usual. For example, \DIFdelbegin \DIFdel{Boris }\DIFdelend \DIFaddbegin \DIFadd{someone }\DIFaddend should recognize
that agreeing to be an expert witness increases the odds that \DIFdelbegin \DIFdel{he }\DIFdelend \DIFaddbegin \DIFadd{they }\DIFaddend will be
targeted, and should be more careful about what he puts into email while
preparing and delivering his testimony.

The corollary to situational awareness is to de-escalate when you can.  Being on
guard all the time is exhausting and quickly leads to \textbf{security fatigue}
\cite{Stan2016}. If you are too tired to follow good practices, knowing them does
you no good.

\DIFdelbegin %DIFDELCMD < \rulemajor{Rule 3: Use relevant threat models.}
%DIFDELCMD < %%%
\DIFdelend \DIFaddbegin \section*{\DIFadd{Rule 3: Use relevant threat models.}}
\DIFaddend 

Edward Snowden and the journalists who worked with him took extraordinary
measures to safeguard themselves against \textbf{state-level actors}
\cite{Snow2019}, but most of us aren't involved in issues of national
security and don't need to take those kinds of precautions. Instead, we
typically face one of three kinds of threat illustrated by the examples below.

\begin{itemize}
\DIFdelbegin %DIFDELCMD < 

%DIFDELCMD < %%%
\DIFdelend \item
  \textbf{Casual threats} are opportunistic. For example, Monica, a professor in
  psychology, is targeted by Mohan, an undergraduate in computer science who
  spends hours every day in online echo chambers complaining about how ``SJW
  bullshit'' is ruining tech. He really didn't enjoy Monica's guest lecture on
  discrimination and inclusivity in his software engineering class, and thinks
  it would be a laugh to make her the target of anonymous abuse online. He is
  unlikely to invest significant effort in his attack (at least not initially),
  but his attack may be backed up by more knowledgeable \DIFdelbegin \DIFdel{advisors in }\DIFdelend \DIFaddbegin \DIFadd{members of }\DIFaddend online
  forums. \DIFaddbegin \DIFadd{They are almost certainly not computer security specialists; instead,
  they are probably older versions of Mohan who have picked up a few tricks and
  bits of software and enjoy the digital equivalent of throwing bricks through
  strangers' windows.
}\DIFaddend 

\item
  \textbf{Intimate threats} come from people who know their targets' passwords
  or have a chance to install spyware on their targets' devices \cite{Leit2019}.
  For example, Elena, graduate student, is targeted by her \DIFaddbegin \DIFadd{former }\DIFaddend romantic
  partner Eric, who is also a graduate student but not in the same department.
  Their relationship had become increasingly abusive over the last two years.
  With the help of friends, Elena has moved out of their shared apartment and is
  rebuilding her life; Eric is obsessed with the idea that she left him for
  someone else and is now stalking her.

\item
  \textbf{Insider threats} come from people who have legitimate access to
  accounts and devices. For example, Boris, professor of medicine, is targeted
  by Bethany, who works for the university's IT department.  Boris has agreed to
  serve as an expert witness in an upcoming liability case involving a large
  chemical company; Bethany has been asked by a former colleague to find out
  what he is going to say in order to discredit his testimony.
\DIFdelbegin %DIFDELCMD < 

%DIFDELCMD < %%%
\DIFdelend \end{itemize}

\DIFdelbegin %DIFDELCMD < \rulemajor{Rule 4: Use a password manager.}
%DIFDELCMD < %%%
\DIFdelend \DIFaddbegin \section*{\DIFadd{Rule 4: Use a password manager.}}
\DIFaddend 

Using a weak password is a good way to ensure that your account will eventually
be compromised, in part because \textbf{dictionary attacks} can be run offline
against encrypted password files to find passwords that match common
patterns. Using a clever password scheme, such as the name of the site plus a
word only you know, does not increase security by much: whatever scheme you have
thought of, attackers have seen before. And since people are often identified on
multiple sites by the same email address, as soon as one site where you've used
that scheme is compromised, attackers can guess the scheme and use it elsewhere.

Reusing passwords ensures that damage spreads, so using a different password for
each site helps limit harm if any are compromised. However, strong passwords are
hard to remember and to type, so always use a \textbf{password manager} that
generates strong passwords and saves them all under a master \textbf{passphrase}.
Your passphrase should be several words long and something you are unlikely to
forget. It does create a single point of attack, but is still safer than
choosing passwords yourself, since password managers aren't fooled by
similar-seeming sites like \DIFdelbegin \texttt{\DIFdel{paypaI.com}}%DIFAUXCMD
\DIFdel{.}\DIFdelend \DIFaddbegin \DIFadd{paypaI.com.
}\DIFaddend 

\DIFdelbegin \DIFdel{And despite what you may have heard,
writing }\DIFdelend \DIFaddbegin \begin{quote}
  \DIFadd{Writing }\DIFaddend passwords down and keeping them in your wallet is not \DIFdelbegin \DIFdel{a bad practice:
you have been keeping }\DIFdelend \DIFaddbegin \DIFadd{necessarily a
  bad practice---it depends on who is doing it. For example, an elderly person
  who finds tech confusing might well choose simple, easy-to-guess passwords for
  their accounts if they have to be remembered. On the other hand, they have a
  lifetime of practice keeping track of }\DIFaddend bits of paper\DIFdelbegin \DIFdel{safe since you were a child, and you'll know if your written passwords go missing or are stolen.
}\DIFdelend \DIFaddbegin \DIFadd{, and will probably notice
  if their purse or wallet is stolen\ldots{}
}\end{quote}
\DIFaddend 

\DIFdelbegin %DIFDELCMD < \rulemajor{Rule 5: Use two-factor authentication.}
%DIFDELCMD < %%%
\DIFdelend \DIFaddbegin \section*{\DIFadd{Rule 5: Use two-factor authentication.}}
\DIFaddend 

Authentication relies on something you \emph{know} (like a password), something
you \emph{have} (like a security key), or something you \emph{are} (like your
fingerprints). \textbf{Two-factor authentication} requires two of these together
to establish your identity, e.g., a password (which can be stolen
electronically) plus a random code generated by an app on your phone (which
means attackers need access to you).

2FA is as important to security as using a password manager, but where possible,
you should rely on an app for 2FA instead of using text messages. What you
should \emph{never} do is share a confirmation code, since a common attack is to
trigger a password reset and then call the victim pretending to be from the IT
department and ask them to read the code back to ``verify'' your account. As
soon as you do this, the attacker can change your password and get into your
account.

\begin{quote}
  Many security experts now recommend using a physical 2FA key such as a
  YubiKey, which fits on a keychain and plugs into a standard USB port.  Sites
  like \href{https://techsolidarity.org/}{Tech Solidarity} have easy-to-follow
  instructions explaining how to set them up for a range of popular social
  networking sites.
\end{quote}

\DIFdelbegin %DIFDELCMD < \rulemajor{Rule 6: Think before opening.}
%DIFDELCMD < %%%
\DIFdelend \DIFaddbegin \section*{\DIFadd{Rule 6: Think before opening.}}
\DIFaddend 

Much of the software we use was designed in more innocent times, and since
companies are almost never held liable for the damage caused by their software,
they have consistently prioritized convenience for the many over harm to the
few. One common example is documents that contain code called ``macros'' that
automatically execute when the document is opened. Used for good, a macro can
check that an address field has been filled in correctly. Used for evil, it can
\DIFdelbegin \DIFdel{mail the contents of }\DIFdelend \DIFaddbegin \DIFadd{email everyone in }\DIFaddend your address book\DIFdelbegin \DIFdel{anywhere }\DIFdelend \DIFaddbegin \DIFadd{, or send a copy of those addresses to anyone
}\DIFaddend in the world.  Microsoft Word and Excel are particularly notorious for this
vulnerability, but many other kinds of documents have the same flaw.

Attempts to get you to open an email attachment, click on a link, install
software, or log into a website are called \textbf{phishing} attacks. The
strongest defense is to never do these things, but in the modern world that
would make most work impossible. The second-best defense is to take sensible
precautions. \DIFdelbegin \DIFdel{For example,
}\emph{\DIFdel{never}} %DIFAUXCMD
\DIFdel{open email attachments without first running them through a virus scanner---not even
if you trust the person who sent it, because their computer might just have been compromised.
Using cloud applications such as
Google Docs instead of desktop applications means you don't have to download files, but the tradeoff is that you are now giving all of
your data to a company
whose business model is to sell your personal information to unaccountable third parties}\DIFdelend \DIFaddbegin \DIFadd{If you are able, invest in virus scanning software such as
}\href{https://www.proofpoint.com/us/products/email-protection}{Proofpoint}
\DIFadd{to scan email attachments before you download them. While many email clients
have virus scanning technology built-in, this will offer an extra layer of
protection}\DIFaddend .

Similarly, don't click links in emails without checking them first: instead,
hover over the link and see if it matches the site it claims to be.
Alternatively, log into the site manually rather than following the provided
link. It takes more time, but is still faster than fixing your credit rating.
And when you go to a web site, check the real domain name in the URL: \DIFdelbegin \texttt{\DIFdel{paypaI.com}} %DIFAUXCMD
\DIFdelend \DIFaddbegin \DIFadd{paypaI.com
}\DIFaddend with an upper-case ``I'' instead of a lower-case ``l'' is not the site it
pretends to be, and \texttt{wwwpaypal.com} is a different domain than
\texttt{www.paypal.com}.

\begin{quote}
  Many sites send an email with a random URL to confirm your identity when you
  are resetting your password. On the one hand, this means that an attacker has
  to get access to your email in order to break into your account. On the other
  hand, random URLs are hard to type in, so these emails encourage us to click
  on links in emails. If you are not expecting a password reset
  email, \emph{don't click on the link}.
\end{quote}

While phishing attacks are wide-ranging, \textbf{spearphishing} \DIFdelbegin \DIFdel{uses }\DIFdelend \DIFaddbegin \DIFadd{describes the
use of }\DIFaddend data harvested from previous victims to attack specific targets. Here,
the best defense is to very suspicious emails, e.g., by phoning people to
confirm their identity. It's particularly important to do this when you are sent
things like password reset instructions. Many IT departments send out messages
that are indistinguishable from spearphishing attacks, which just trains people
to be victims.

\DIFdelbegin %DIFDELCMD < \rulemajor{Rule 7: Erase before discarding.}
%DIFDELCMD < %%%
\DIFdelend \DIFaddbegin \section*{\DIFadd{Rule 7: Erase before discarding.}}
\DIFaddend 

Moving files into the trash and then emptying it does not actually erase the
data: it just tells the computer that the space is available for \DIFdelbegin \DIFdel{re-use}\DIFdelend \DIFaddbegin \DIFadd{reuse}\DIFaddend . (This is
why reporters and private investigators regularly go dumpster diving.) The best
way to address this problem is to encrypt \DIFdelbegin \DIFdel{the }\DIFdelend \DIFaddbegin \DIFadd{your hard }\DIFaddend drive, which is a quick
setup option for all major operating systems these days.

Even with that, you should act as if any device you throw away is going to fall
into unfriendly hands. Use a secure deletion tool like BleachBit \DIFdelbegin \DIFdel{on }\DIFdelend \DIFaddbegin \DIFadd{(}\DIFaddend Linux or
Windows\DIFaddbegin \DIFadd{) or FileShredder (MacOS) }\DIFaddend before selling, recycling, or discarding your
hardware, but keep in mind that this doesn't affect backups or files stored
online on sites like Dropbox. And keep in mind that it is practically impossible
to truly delete data from social networking sites: in most cases, their
``delete'' usually means ``don't show any more'' rather than ``erase all past
record of''.

\DIFdelbegin %DIFDELCMD < \rulemajor{Rule 8: Check your devices and accounts periodically.}
%DIFDELCMD < %%%
\DIFdelend \DIFaddbegin \section*{\DIFadd{Rule 8: Check your devices and accounts periodically.}}
\DIFaddend 

\DIFdelbegin \DIFdel{Google, Facebook, and others make money }\DIFdelend \DIFaddbegin \DIFadd{Many tech companies who offer free products and services make money by }\DIFaddend selling
targeted advertising to you \DIFdelbegin \DIFdel{and data }\DIFdelend \DIFaddbegin \DIFadd{using the data they have }\DIFaddend about you.  They \DIFdelbegin \DIFdel{have been forced to give
users a semblance of }\DIFdelend \DIFaddbegin \DIFadd{do give
users some }\DIFaddend control over personal data, but they frequently change their terms of
service in opaque ways. Seemingly-innocuous information can give attackers
valuable clues: restaurant ``likes'' reveal where you were at specific times,
while funny stories about childhood birthday parties reveal likely answers to
security questions. \DIFdelbegin \DIFdel{Take advantage of what little protection they have been forced to allow you;
again, if you }\DIFdelend \DIFaddbegin \DIFadd{Again, it's a good practice to }\DIFaddend get into the habit of
\DIFdelbegin \DIFdel{doing this }\DIFdelend \DIFaddbegin \DIFadd{checking your privacy settings }\DIFaddend every time you do some other regular task\DIFdelbegin \DIFdel{,
you're more likely to do both}\DIFdelend .

Unfortunately, even if you do this, information may leak through \DIFdelbegin \DIFdel{others}\DIFdelend \DIFaddbegin \DIFadd{other
means}\DIFaddend . For example, attackers can friend your friends in an attempt to get
information about you, such as the name of your first school. And as bad as
\DIFdelbegin \DIFdel{major }\DIFdelend social media sites are \DIFaddbegin \DIFadd{for social engineering in this way}\DIFaddend , cell phone
applications are often worse (not even counting the ones that turn out to be
government-sponsored spyware \cite{Schn2019}). \DIFdelbegin \DIFdel{If }\DIFdelend \DIFaddbegin \DIFadd{In general, if }\DIFaddend a game wants
access to your camera and address book, you should probably find a different
game to play.

Since social media is a fact of life for most of us, you should check your
settings periodically, just as you would take your car in for an oil change.
(\DIFdelbegin \DIFdel{In fact,
the author does }\DIFdelend \DIFaddbegin \DIFadd{The authors do }\DIFaddend these things at the same time in order to remember both.)  Turn
off everything you can and then use a tracking blocker such as Ghostery to
reduce information leakage.

Many experts recommend using separate devices or accounts for work and personal
life, but this is increasingly unrealistic. Everyone checks their personal email
from their work device eventually, and everyone uses their personal phone for
2FA. However, you should consider getting a second phone for international
travel: the legalities around who can take your devices and/or force you to
unlock them are complicated and frequently overstepped, so you should assume
that anything on or connected to the devices you are traveling with will be
compromised.

\begin{quote}
  \emph{Never} plug a random USB drive into your device: it's like letting a
  complete stranger into your home unsupervised.
\end{quote}

\DIFdelbegin %DIFDELCMD < \rulemajor{Rule 9: Fight back.}
%DIFDELCMD < %%%
\DIFdelend \DIFaddbegin \section*{\DIFadd{Rule 9: Fight back.}}
\DIFaddend 

Casual attackers may eventually get bored and move on, but like all bullies,
they will also often revisit previous victims, and even if they don't, they are
likely to pick new ones. If you have been attacked:

\begin{enumerate}
\DIFdelbegin %DIFDELCMD < 

%DIFDELCMD < %%%
\DIFdelend \item
  \textbf{Find support.} Being targeted is frightening and wearying,
  particularly if you belong to one of the many groups that are targeted in real
  life as well as online. Let family, friends, and colleagues know what is
  happening so that they can support you. They may also be able to offer advice
  if they have been in similar situations.

\item
  \textbf{\DIFdelbegin \DIFdel{Document everything.}\DIFdelend \DIFaddbegin \DIFadd{Use anti-harassment apps}\DIFaddend }
  \DIFaddbegin \DIFadd{like }\href{https://www.blockpartyapp.com/}{{Block Party}} \DIFadd{and document
  everything. }\DIFaddend Save emails and take screenshots of sites like Facebook and
  Twitter (in case attackers delete or alter material).

\item
  \textbf{Do not engage directly.} Casual attackers are often seeking attention,
  so a direct response often encourages further attacks (and can draw attention
  from like-minded attackers).

\item
  \textbf{Report the attack.} Social media sites have done everything they can
  to avoid \DIFdelbegin \DIFdel{being accountable for facilitating }\DIFdelend \DIFaddbegin \DIFadd{legal accountability for }\DIFaddend online attacks, but companies and
  universities will usually take what steps they can once they know there is a
  problem. In the \DIFdelbegin \DIFdel{author' s }\DIFdelend \DIFaddbegin \DIFadd{authors' }\DIFaddend experience, they are \DIFdelbegin \DIFdel{most likely to do this }\DIFdelend \DIFaddbegin \DIFadd{more inclined to take real
  action against the attacker }\DIFaddend if they believe that you might speak publicly
  about what has happened and thereby damage their reputation, so never agree to
  a non-disclosure agreement that would prevent you from doing so.
\DIFdelbegin %DIFDELCMD < 

%DIFDELCMD < %%%
\DIFdelend \end{enumerate}

\DIFdelbegin %DIFDELCMD < \rulemajor{Rule 10: It's not all about you.}
%DIFDELCMD < %%%
\DIFdelend \DIFaddbegin \section*{\DIFadd{Rule 10: It's not all about you.}}
\DIFaddend 

Our final rule brings us full circle to the first one. We don't just wear masks
to prevent ourselves from becoming infected: we also wear them so that we will
not infect others. Similarly, if you do not take precautions with online
security then you are putting others at risk.  Simple steps like putting
passwords on PDFs that contain sensitive information can go a long way to deter
attackers, in the same way that a sturdy-looking bike lock encourages would-be
thieves to go after some other bike. And if you \emph{are} compromised, let
those affected know as soon as you can.

\DIFdelbegin \DIFdel{But the }\DIFdelend \DIFaddbegin \DIFadd{The }\DIFaddend only long-term way to improve everyone's online safety is to pressure
politicians to strengthen liability legislation so that companies, universities,
and other institutions have real incentives to take meaningful action. Cars and
drugs are as safe as they are because their manufacturers are liable for
negligence and harm. The sooner software companies and social media sites are
liable as well, the safer all of us will be.

\DIFdelbegin %DIFDELCMD < \begin{quote}
%DIFDELCMD <   %%%
\DIFdel{The author is }\DIFdelend \DIFaddbegin \subsection*{\DIFadd{Acknowledgments}}

\DIFadd{The authors are }\DIFaddend grateful to Claire Bowen (Urban Institute), Leigh Honeywell
(Tall Poppy), Mike Hoye (Mozilla), Scott Jackson (RStudio), Rick Johnson
(RStudio), Cheng Soon Ong (CSIRO), and Sydney Young (EFF) for their feedback on
this article.
\DIFdelbegin %DIFDELCMD < \end{quote}
%DIFDELCMD < %%%
\DIFdelend 

\bibliography{10-safety}

\section*{Appendix: VPNs and Tor}

A Virtual Private Network (VPN) connects your device to a server, then has the
server make connections to other machines on your behalf. All messages between
your device and the server are encrypted, and the server can be managed by
professional IT staff in a jurisdiction with tight privacy laws to increase your
safety. A specialized web browser called Tor routes messages randomly through a
network of servers, making traffic much harder to track. Both of these reduce
risk, but neither eliminates it if your device has been compromised, if the VPN
is compromised (or subpoenaed), or if you log in to accounts over Tor (thereby
revealing your identity to those sites).

\end{document}
